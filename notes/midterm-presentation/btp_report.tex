\documentclass{beamer}
 
\usepackage[utf8]{inputenc}
 
 
%Information to be included in the title page:
\title{Formally verified hardware synthesizer}
\author{Vismay Raj T}
\institute{IIT Palakkad}
\date{2019}
 
 
 
\begin{document}
 
\frame{\titlepage}
 
\begin{frame}
    \frametitle{Overview}
    \begin{itemize}
        \item What is formal verification
        \item Dependent types
        \item Coq
        \item How to take this to hardware
        \item Future plans
    \end{itemize}
\end{frame}

\begin{frame}
    \frametitle{What is formal verification}
    \textit{Have we made what we were trying to make?} \\~\\ 
    Formal verification is the process of proving or disproving the correctness of intended algorithms
    with respect to a certain \textbf{formal specification} or property \\~\\
    The verification of these systems are done by providing a \textbf{formal proof} on an abstract 
    \textbf{mathematical model} of the system
    \vfill
\end{frame}

\begin{frame}
    \frametitle{Formal specification}

    \textit{Formal specifications describe what a system should do, not how the system should do it}\\~\\

    Formal specifications are used to describe a system , analyse it's behavior , and it also aid in the 
    design of the same by verifying the above stated properties through reasoning tools.

    \vfill

\end{frame}

\begin{frame}[fragile]
    \frametitle{Dependent Types}

    A Dependent types is a type whose definition depends on a value. It is an overlapping feature of type 
    theory and type systems. Consider.
    \begin{verbatim}
    Function Concat 
    (a: 'a list n) (b: 'a list m) : 'a list (n+m) 
        := a::b
    \end{verbatim}
    similarly in Dependent types we can also have functions that return types. \textit{Coq is an example}
    \begin{verbatim}
        Definition foo (n:nat) : type :=
            match n with
            | 0 => nat
            | _ => (nat*nat)%type
            end.
    \end{verbatim}
\end{frame}

\begin{frame}[fragile]
    \frametitle{Why dependent types}
    Now lets see how a function that return a type will be useful. Consider the following code segment
    \begin{verbatim}
        Definition bar (n:nat) : foo n :=
        match n with
        | 0 => 42
        | _ => (42,42)
        end.
    \end{verbatim}
    Now we are able to bring an additional constraint on the value  that is being returned , and that 
    constraint depends on the \textbf{value} of the input. \\~\\
    And the best part is that this constraint is checked during compile time.!!
    
\end{frame}

\begin{frame}
    \frametitle{Coq}
        Coq is an interactive theorem prover. It allows the expression of mathematical assertions,
        mechanically checks the proofs of these assertions and helps in extracting certified program 
        \\~\\
        \begin{itemize}
            \uncover<2->{\item Has features of a functional programming language}
            \uncover<3->{\item Dependent types allows constraints over the input and output of functions}
            \uncover<4->{\item Being a theorem prover it helps in proving that the written function is behaving as expected}
        \end{itemize}
    

\end{frame}

\begin{frame}[fragile]
    \frametitle{An example of a stack machine}
    The process involves . 
    \begin{itemize}
        \uncover<1->{\item Defining the type for an expression}
        \uncover<2->{\item Defining a meaning for the expression (ie, how it is evaluated)}
        \uncover<3->{\item Defining a program as a list of instructions}
        \uncover<4->{\item Defining meaning to a program (ie, how the list of instructions are executed)}
        \uncover<5->{\item Defining a compiler which will take an expression and returns a list of instructions which is the program}
    \end{itemize}
\end{frame}

\begin{frame}[fragile]
    \frametitle{Continues...}
    Now we have , 
    \begin{itemize}
        \uncover<1->{\item A type for expression}
        \uncover<2->{\item A way to calculate $exp \rightarrow nat$}
        \uncover<3->{\item A way to make a program $exp \rightarrow program$}
        \uncover<4->{\item A way to run the program and get an output $program \rightarrow nat liost$}
    \end{itemize}

    \uncover<5->
    {
        Now what we need to verify is the correctness of the compiler.
        
        $ top(runProg(compileExp \;\; e)) = n \Longrightarrow (CalcExp \;\; e) = n$
    }
    

\end{frame}



\end{document}

