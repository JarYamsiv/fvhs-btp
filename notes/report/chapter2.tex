\chapter{Curry-howard Correspondence}

In programming language theory and proof theory, the Curry–Howard correspondence 
(also known as the Curry–Howard isomorphism or equivalence) is the direct relationship between computer 
programs and mathematical proofs. It states that \textit{"how are type to programs in a functional programming
paradigm , the same relation can be seen between propositions to proofs in a proof system"}.
In other words, the Curry–Howard correspondence is the observation that two families of seemingly unrelated 
formalisms—namely, the proof systems on one hand, and the models of computation on the other—are in 
fact the same kind of mathematical objects. 

\section{Types as propositions}

    The datatypes in a functional programming language can be seen as propositions in a proof system. And hence
the type constructors in functional programming will correspond to relational operations in proof systems.\\
eg:\\
Consider the following functional programming code 
\begin{verbatim}
    val x:int = 3
    val y:string = "hello"
    val z:int*string = (x,y)
\end{verbatim}

Now for this code , there are three variables of which the third one is constructed from the other two.
the existence of a variable of a certain type in a code can be considered as a proposition "x" being true in
a type system. the same goes for the variable "y" . but when we get back to "z" which was constructed from 
both "x" and "y" and it requires the existence of both variables. In a proof system this translates to the function
"and" which will only be true when both the propositions that it has are true ( ie , they exists inside the system)
        
\section{Further relations}
\begin{itemize}

  \item Tuple type \verb|(int,string)|\\
      Creating : \verb| val x : int*string = (4 , ``Hello'')|\\
      Using : \verb| val y : int = #1 x|\\
      
  \item Function type \verb| int -> string |\\
      Creating : \verb| fun f (x:int) :string = ``the value of x is'' ^ toString(x) |\\
      Using : \verb| val y: string = f 45|\\
    
  \item Disjunction type \verb| datatype X = Left of int | $|$  \verb| Right of string |\\
      Creating : \verb| val x:X = Left 4|\\
      Creating : \verb| val y:X = Right ``Hello'' |\\
      Using : \verb| fun f (x:X) :bool = case x of |\\
            \verb|                     Left y = ( y > 0 )|\\
              \verb|                     Right _ = false|\\
      the above given is a function that takes an argument of type X and returns a value of type bool\\
              
  \item Unit type \verb| type t = unit |\\
      Creating : \verb| val x:unit = () |\\
              
\end{itemize}

\section{Conclusion}

    And as we have seen the relation between proofs and programs , we can see that if a program were to infer
the types correctly that means that the corresponding proof of it will also hold true in the system.