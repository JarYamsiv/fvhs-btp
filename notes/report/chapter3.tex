\chapter{Dependent types}
     Dependent types are types whose definition depends on a value, it is an overlapping feature of \textit{type
theory} and \textit{type systems}. Dependent types are used to encode logic's quantifiers like \textit{forall}
 and \textit{there exists}. In functional programming languages like \textit{Agda , ATS , Coq , F* , Epigram}
and \textit{Idris} dependent types may help reduce bugs by enabling the programmer to assign types that 
further restrain the set of possible implementations. 

\section{How do they look like}
    Let's see a few examples. I'll use Coq for the examples. they are easy to understand even if you are
not familiar with Coq. It requires basic understanding of functional programming.\\
    Let's define a function in Coq that takes a natural number and returns a type. This is valid because 
in dependently typed languages types are also treated as values of some other types, thus giving arise to 
the type hierarchy  that can be seen below.
\begin{verbatim}
    Definition foo (n:nat) : type :=
        match n with
        | 0 => nat
        | _ => (nat*nat)%type
    end.
\end{verbatim}

This here is a definition of a function in Coq. This function has an argument \verb|n| that is a natural
number represented by \verb|n:nat|. and the return type of the function is given as \verb|type|. Which
means that what the function returns is a type. \textit{type is a value of type type. There are hierarchy
of types which can be considered as type1 type2 etc.. but they are not visible in the programming\ level}\\
\textit{note: the \% type is to distinguish that the given is a tuple type and not multiplication}\\ \\
But how will this function be useful? \\ Let's consider another function
\begin{verbatim}
    Definition bar (n:nat) : foo n :=
    match n with
    | 0 => 42
    | _ => (42,42)
\end{verbatim}
Now let's see what's happening with this function. Here the input like last time is a natural number n.
and the outputs type is not some constant value instead it's written as \verb|foo n| , which means 
what the function returns has a type that is dependent upon the value of the argument.And looking closely
into the function it can be seen that the function return a single natural number for the input 0 and
a tuple for any other input. This follows from the definition of \verb|foo| which states that the type
needs to be a \verb|nat| for \verb|0| and \verb|nat*nat| for anything else. This way of defining and using
types combined with the strict nature of such programming languages will help the programmer in detecting
and dealing with many of the bugs during compile time itself.

