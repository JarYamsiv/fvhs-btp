\chapter{Asynchronous Circuits}
An asynchronous circuit, or self-timed circuit, is a sequential digital logic circuit 
which is not governed by a clock circuit or global clock signal. 
Instead it often uses signals that indicate completion of instructions and operations, 
specified by simple data transfer protocols.

\section{Important features}
\begin{itemize}
    \item In asynchronous circuits, there is no clock signal, 
    and the state of the circuit changes as soon as the inputs change.
    \item they can be faster than synchronous circuits,
     and their speed is theoretically limited only by the propagation delays of the logic gates
    \item circuit can be sensitive to the relative arrival times of inputs at gates.
\end{itemize}

\section{Circuit Delays in VHDL signals}
\begin{itemize}
    \item \textbf{Transport} - The propagation delay in the system
    \item \textbf{Inertial} - propagation delay and minimum pulse width
    \item \textbf{Delta} - the default if no delay time is explicitly specified.
\end{itemize}
\par We can see that in combinational circuits the outputs are not evaluated instantly when the input changes
There are different types of delays that determine when the changes will occur and how the changes will
occur. As the circuit grows larger such relations become complicated. And verifying the correctness of 
such circuits become important. 

\par The project aims to achieve the formal verification of asynchronous circuits through theorem
prover "Coq"