\chapter{Conclusion and Future Work}

\par During the first half of this semester I was able to understand the working of the proof assistant Coq.
And was also able to write a few proofs using it. And using the book \textit{Certified Programming with
Dependent types} I was able to see "how to specify an algorithm for a machine formally" and also "how to
prove the correctness of it".
\par During the second half of the semester more time was spent to study Coq. And during this time it 
has also decided to dive deeper into formal specifications of Asynchronous circuits.
hardware verification using coq has already been done in \cite{kami} . This project focuses on
Asynchronous circuits.

\section{future works}

\begin{itemize}
    \item Further more on formal specifications\\
        Need to know about more examples of formal specifications and corresponding proof of correctness.
        The book provides the required examples.
    \item VHDL\\
        More insight to writing VHDL codes are required. This is expected to be the target language when
        The specifications are formally verified.
    \item Asynchronous circuits\\
        Since formal verification of hardware is an already explored topic, it is decided to dive deep 
        into formal verification of Asynchronous circuits.
\end{itemize}
