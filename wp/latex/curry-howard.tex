\documentclass{article}
\begin{document}

\section{What is curry-howard correspondence?}.\\
In programming language theory and proof theory, the Curry-Howard correspondence (also known as the Curry-Howard
isomorphism or equivalence, or the proofs-as-programs and propositions- or formulae-as-types interpretation) is the
direct relationship between computer programs and mathematical proofs.\\ \\

\subsection{type constructions in FP}
But before getting into that let's see type constructions in functional programming \textit{Here I'll use SML as an example}

\begin{itemize}
\item Tuple type \verb|(int,string)|\\
  Creating : \verb| val x : int*string = (4 , ``Hello'')|\\
  Using : \verb| val y : int = #1 x|\\
\item Function type \verb| int -> string |\\
  Creating : \verb| fun f (x:int) :string = ``the value of x is'' ^ toString(x) |\\
  Using : \verb| val y: string = f 45|\\

\item Disjunction type \verb| datatype X = Left of int | $|$  \verb| Right of string |\\
  Creating : \verb| val x:X = Left 4|\\
  Creating : \verb| val y:X = Right ``Hello'' |\\
  Using : \verb| fun f (x:X) :bool = case x of |\\
  	  \verb|                     Left y = ( y > 0 )|\\
          \verb|                     Right _ = false|\\
  the above given is a function that takes an argument of type X and returns a value of type bool\\
          
\item Unit type \verb| type t = unit |\\
  Creating : \verb| val x:unit = () |\\
          
\end{itemize}

\subsection{From types to propositions}
But how does the types defined just translate themselves into propositions?\\
consider the sml code \verb|val x:t = ....|. now what does this mean? If this code manages to compile without error then
there the type t exists in the system.\\
Let's denote this proposition by $CH \left( t \right) $ which means that code has a value of type t or the type t exists in the
system.\\
But that was just for a type t what about all the other things that we have discussed above?

\begingroup
\setlength{\tabcolsep}{10pt} % Default value: 6pt
\renewcommand{\arraystretch}{1.5} % Default value: 1
\begin{tabular}{c c c}
  Type & Proposition & Short Notation \\
  \hline
  t   & CH(t) & t \\
  \hline
  (a,b) & CH(a) and CH(b) & A $\wedge$ B ; A $\times$ B\\
  \hline
  left a $\mid$ right b & CH(a) or CH(b) & A $\vee$ B ; A $+$ B\\
  \hline
  a $\to$ b & CH(a) implies CH(b) & A $\Longrightarrow$ B\\
  \hline
  unit & true & 1
\end{tabular}
\endgroup                

\begin{itemize}
\item type parameter \verb| 'a | can be considered as $ \forall a $
\item consider the function \verb| fun f (x: 'a) : 'a * 'a = (x,x) |
  this is logically equivalent to $ \forall a \; \; \; a \Longrightarrow a \wedge a $

\item The elementary proof task is represented by a sequent. Notation $ A,B,C \vdash G $ the premises are A,B and C and the
  goal is G.
\item proofs are achieved through axioms and derivation rules. Axioms : this sequent  is true. Derivation: if such sequents are
  true then other such sequents are true.
\item consider the SML expression \verb| val x = 3; val a = Int.toString(x:int) ^ ``abc''; | an expression of type String
  This is equivalent to the sequent $ Int \vdash String $. This sequent uses the fact that it uses the already computed expression
  x which is an integer. and hence that becomes the premise and what we are computing here is the value of string which
  becomes the goal.
\item Now consider the function \verb| fun f (x:int) :string = Int.toString(x) ^ ``abc'' |. this takes an argument of integer
  type and returns a value of string type.This doesn't assume that x is defined outside. this uses x as an argument . and
  it is represented by the sequent $ \phi \vdash Int \Rightarrow String $. The empty set on the premises implies that this
  expression doesn't use any variables that are previously computed. and the goal is the function type.
\item sequents only describe the types and doesn't say anything about the value that is being computed.
\end{itemize}

Now lets see how the type constructions stated above translates into axioms and derivation rules.

\begin{itemize}

\item the tuple type. $ A \times B $\\
  Creation : $ A, B \vdash A \times B $ . that is if we have types A and B then we have type $ A \times B $.\\
  Usage: $ A \times B \vdash A $ or $ A \times B \vdash B $ .
\item Function type . $ A \Rightarrow B $.\\
  Creation : If $ A \vdash B$ then we have $ \phi \vdash A \Rightarrow B $.\\
  Usage : $ A \Rightarrow B , A \vdash B $.

\item Disjunction type $ A + B $:\\
  Creation : $ A \vdash A + B $ and also $ B \vdash A + B $\\
  Usage : $ A+B , A \Rightarrow C , B \Rightarrow C \vdash C $ . this corresponds to the function (that uses case statements )
  that we've written in   the SML example for disjunction.\\

\item Unit type 1:\\
  Create : $ \phi \vdash 1 $\\
  
  
\end{itemize}


\end{document}